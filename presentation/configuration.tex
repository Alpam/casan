\documentclass [12pt] {beamer}

\usepackage [utf8] {inputenc}
\usepackage {helvet}

\usepackage {graphicx}
\usepackage{verbatim}

% \usepackage {tikz}

% \usepackage {hyperref}
\hypersetup
{
pdfauthor={Philippe PITTOLI},
	pdfsubject={A short presentation of the arduino capabilities},
pdftitle={A short presentation of the arduino capabilities},
pdfkeywords={Arduino, Electronic}
}


\usetheme {default}
% \useoutertheme [headline=empty,footline=authortitle] {miniframes}
\usefonttheme {structurebold}

\setbeamercolor {frametitle} {fg=red}
\setbeamerfont {frametitle} {series=\bfseries}
\setbeamertemplate {frametitle} {
\textbf {\insertframetitle} \par
}

\pgfdeclareimage[height=0.5cm]{logo}{img/icube}
% \logo{\pgfuseimage{logo}}

\setbeamertemplate {footline} {%
\vskip 1ex
	\hskip 1ex
	% \insertlogo
	\pgfuseimage {logo}
	\hfill
	\insertframenumber/\inserttotalframenumber
	\hspace* {1ex}
\vskip 1ex
}

\setbeamertemplate {navigation symbols} {}

\title {A pragmatic approach \\
	of the arduino}
\author {Philippe PITTOLI \\ \texttt {philippe.pittoli@etu.unistra.fr}}
\institute {ICube -- Network Research Group}
\date [Network Research Group meeting] {9 july 2013}

\newcommand {\imply} {$\Rightarrow$\ }

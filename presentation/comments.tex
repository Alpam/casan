% what is to say : 

% What is Arduino

% Arduino is a tool for making computers that can sense and control the physical world. It's an open-source physical computing platform based on a simple microcontroller board, the electronic behind the arduino is quite simple, you can do an arduino yourself with some pieces and developing on it is very simple, it's C++ code with an development environment.

% Arduino can be used to develop interactive objects, taking inputs from a variety of switches or sensors, and controlling a variety of lights, motors, and other physical outputs. Arduino projects can be stand-alone, or they can be communicate with software running on your computer.

% Why Arduino 

% There are many other microcontrollers and microcontroller platforms available for physical computing. Parallax Basic Stamp, Netmedia's BX-24, Phidgets, MIT's Handyboard, and many others offer similar functionality. All of these tools take the messy details of microcontroller programming and wrap it up in an easy-to-use package. Arduino also simplifies the process of working with microcontrollers, but it offers some advantage for teachers, students, and interested amateurs over other systems:

% Inexpensive - Arduino boards are relatively inexpensive compared to other microcontroller platforms. The least expensive version of the Arduino module can be assembled by hand, and even the pre-assembled Arduino modules cost less than $50

% Cross-platform - The Arduino software runs on Windows, Macintosh OSX, and Linux operating systems. Most microcontroller systems are limited to Windows.

% Simple, clear programming environment - The Arduino programming environment is easy-to-use for beginners, yet flexible enough for advanced users to take advantage of as well. For teachers, it's conveniently based on the Processing programming environment, so students learning to program in that environment will be familiar with the look and feel of Arduino

% Open source and extensible software- The Arduino software is published as open source tools. You can add AVR-C code (an C implementation for embedded code, on ATmega controllers) directly into your Arduino programs if you want to.

% Open source and extensible hardware - The Arduino is based on Atmel's microcontrollers. The plans for the modules are published under a Creative Commons license.



% the ~ symbols are to indicate that we can simulate analog output with these
% digital pins by setting power on and off quickly

% ...
% and there is a lot more components we can get
%
%  9v battery snap
%  Potentiometer 10kilohm
%  alphanumeric LCD

% we can modify this electronic component as we modify an opensource code
% no pre-require knowledge


% snippets
\begin {columns} [c,onlytextwidth]
	\begin {column} [c] {.6\textwidth}
	   \begin {center}
		\includegraphics [width=.9\textwidth,keepaspectratio]{delicious}
	   \end {center}
	\end {column}
	\pause
	\begin {column} [c] {.4\textwidth}
	   Delicious has a RESTful API
	\end {column}
\end {columns}

{
	\small
	\hspace* {5mm}
	\alert {GET http://feeds.delicious.com/v2/json/pdagog/histinfo}
}
\vspace* {3mm}

\pause
Result (JSON format): \\
\begin {quote}
	\scriptsize
	[ \\
		\hspace* {3mm} \{ \\
		\hspace* {6mm} "a": "pdagog", \\
		\hspace* {6mm} "d": "Google60", \\
		\hspace* {6mm} "u": "http://www.masswerk.at/google60/", \\
		\hspace* {6mm} "t": ["histinfo", "web", "humour"] \\
		\hspace* {3mm} \}, \\
		\hspace* {3mm} \{ \\
		\hspace* {6mm} "a": "pdagog", \\
		\hspace* {6mm} "d": "Unix Preservation society", \\
		\hspace* {6mm} "u": "http://minnie.tuhs.org/", \\
		\hspace* {6mm} "t": ["freebsd", "unix", "histinfo"] \\
		\hspace* {3mm} \}, ... \\
	]
\end {quote}
